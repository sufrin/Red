%xelatex
% Options for packages loaded elsewhere
\PassOptionsToPackage{unicode}{hyperref}
\PassOptionsToPackage{hyphens}{url}
%
\documentclass[11pt,a4paper]{article}
\usepackage{alltt}
\usepackage{pdffig}
\usepackage{hyperref}
\usepackage{lmodern}
\usepackage{amssymb,amsmath}
\usepackage{unicode-math}
\usepackage{longtable,booktabs}
\defaultfontfeatures{Scale=MatchLowercase}
\defaultfontfeatures[\rmfamily]{Ligatures=TeX,Scale=1}
\setmonofont[]{DejaVuSansMono.ttf}
\setmathfont[]{DejaVuSans.ttf}
%
%
\parindent=0pt\parskip=\medskipamount
\newcommand{\Screen}[2][height=4in]{
 \begin{center}
  \pdffig[#1]{pdf/#2}
   \end{center}}

\newcommand{\ScreenFig}[2]{
\begin{figure}[tb]
\Screen[width=4.0in]{#2}
\caption{#1}\label{#2}
\end{figure}
}

\newcommand{\ScreenFigHere}[2]{
\begin{figure}[h]
\Screen[width=3.5in]{#2}
\caption{#1}\label{#2}
\end{figure}
}
%
%
\title{AppleRed (Version 1.01)}
\author{Bernard Sufrin}
\date{November 24th 2022}

\begin{document}
\maketitle

\hypertarget{applered}{\section{AppleRed}\label{applered}}


\hypertarget{introduction}{\subsection{Introduction}\label{introduction}}

\texttt{Red} (also known as \texttt{AppleRed}) is no-frills
\texttt{UNICODE}-capable \emph{modeless} (and \emph{colourless}) text
editor with a simple implementation that can be customized using the
\texttt{redscript} language, and can be straightforwardly extended using
\texttt{scala}. It has no pretensions to being an all-encompassing
workplace, and unlike many IDE and modern editors does not confuse its
user by spontaneously trying to be helpful: everything it does it does
in response to user input from mouse, pad, or keyboard.

\ScreenFigHere{A Session Window}{Screen1}
\ScreenFigHere{Session Window with tentative selection}{Screen2}

Its \texttt{undo}/\texttt{redo} facility is conventional, but in
addition to this it has a \emph{cut ring} that retains the last 240 or
so \emph{recently deleted} blocks of contiguous text. This feature is
orthogonal to \texttt{undo}/\texttt{redo} and makes it straightforward
to re-insert material that has -- perhaps inadvertently -- been deleted
from any editing session running in the same editor server. The cut ring
interface is similar to an editing session interface, so material can be
transferred from it by copy/paste to any application (including
\texttt{Red}) that has an interface to the system clipboard.

I have used \texttt{Red} mainly for preparing \texttt{(xe-)latex} and
\texttt{markdown} manuscripts, as well as for preparing programs that
don't warrant the burden of getting to grips with an IDE.


Its support for \texttt{latex} manuscripts includes the straightforward
manipulation of nested blocks delimited by
\texttt{\textbackslash{}begin\{...\}} and
\texttt{\textbackslash{}end\{...\}}.

Its support for \texttt{xml} includes the straightforward manipulation
of nested blocks delimited by
\texttt{\textless{}id\ ...\ \textgreater{}} and
\texttt{\textless{}/id\textgreater{}}, as well as
\texttt{\textless{}id\ ...\ /\textgreater{}} constructs.

In both cases, when \texttt{Red/Select} is enabled, clicking at the left
of the opening of a block, or at the right of the closing of a block
will cause that entire block (including nested sub-blocks) to be
selected (``tentatively'').

In addition, clicking at the right of any closing bracket will cause all
text between that bracket and the corresponding opening bracket to be
selected. Dually, clicking at the left of any opening bracket will cause
all text between that bracket and the corresponding closing bracket to
be selected (``tentatively''). The built-in corresponding pairs are:
``\texttt{\{\}}'', ``\texttt{()}'', ``\texttt{{[}{]}}'', as well as most
of the unicode bracketing pairs. Additional pairs can be added
straightforwardly.



\hypertarget{tex-pandoc}{%
\subsubsection{\texorpdfstring{\texttt{Tex},
\texttt{Pandoc}}{Tex, Pandoc}}\label{tex-pandoc}}

The \textbf{\texttt{Tex}} button on the menu bar\footnote{The
configuration provided puts this button on the menu bar when a
\texttt{.tex} document is being edited.} generates \texttt{pdf}
from the (default tex) source manuscript -- usually, but not
necessarily, the manuscript being edited -- using a script
\texttt{redpdf} that is expected to be somewhere on the user's
\texttt{PATH}. The \texttt{redpdf} I use assumes that the manuscript
is in latex format (\texttt{xelatex} format if its first line is
the comment \texttt{\%xelatex}), and invokes \texttt{pdflatex} (or
\texttt{xelatex}) with appropriate parameters before invoking the
platform \texttt{pdf} viewer (\texttt{Skim} on \texttt{OS/X},
\texttt{evince} on \texttt{Linux}).

A little care has to be taken to ensure that the fonts used in
production of the pdf file include gyphs corresponding to all the
Unicode characters appearing in the manuscript. The present xelatex
manuscript uses the following font setup:

\begin{verbatim}
\usepackage{amssymb,amsmath}
\usepackage{unicode-math}
\usepackage{longtable,booktabs}
\defaultfontfeatures{Scale=MatchLowercase}
\defaultfontfeatures[\rmfamily]{Ligatures=TeX,Scale=1}
\setmonofont[]{DejaVuSansMono.ttf}
\setmathfont[]{DejaVuSans.ttf}
\end{verbatim}

The \textbf{\texttt{Pandoc}} button on the menu bar\footnote{When
a \texttt{.md} document is being edited.} translates the manuscript
being edited into \texttt{.pdf} form using the (easily changeable)
script \texttt{redpandoc}.  My personal variant of the script assumes
that the manuscript is in the format:

\begin{verbatim}
        markdown+tex_math_dollars+pandoc_title_block
\end{verbatim}

It uses \texttt{xelatex} as its final \texttt{-\/-pdf-engine}; so you can
use a wide variety of Unicode symbols directly in the manuscript. For
example:

\begin{verbatim}
        $$e^{iπ}+1=0$$
\end{verbatim}

yields: \[e^{iπ}+1=0\]

It's possible, but pretty pointless, to revert to straight
\texttt{pdflatex} as the pdf engine: you just deny yourself the use
of a fuller range of Unicode characters in your markdown manuscript.

My \texttt{redpandoc} script will simultaneously generate an
\texttt{.html} version of the output if a file \texttt{PANDOC-HTML} is
present in the source folder; and will generate pdf \emph{via} the html
file (rather than directly) if a file \texttt{PANDOC-PDF-VIA-HTML} is
present.

To set up the same fonts as used in the present xelatex manuscript
one would use a title block like this at the head of the markdown
manuscript.

\begin{verbatim}
    ---
    title: AppleRed
    fontfamily: DejaVu Sans
    mainfont: DejaVuSerif.ttf
    sansfont: DejaVuSans.ttf
    monofont: DejaVuSansMono.ttf 
    mathfont: DejaVuSans.ttf
    header-includes:
      - '`\usepackage{unicode-math}`{=xelatex}'
    ---
\end{verbatim}

\hypertarget{synctex-features}{%
\subsubsection{\texorpdfstring{\texttt{SyncTex}
features}{SyncTex features}}\label{synctex-features}}

The \texttt{redpdf} script invoked by the \texttt{Tex} button uses
\texttt{synctex} features to collaborate with the platform viewer
(\texttt{Skim} on \texttt{OS/X}, \texttt{evince} on \texttt{Linux}).

\begin{enumerate}
\def\labelenumi{\arabic{enumi}.}
\item
  The current editing position in the manuscript being edited is shown
  as the viewer opens (or reopens) the newly-generated \texttt{pdf} file
  after the \texttt{Tex} button is pressed.
\item
  The current position in the editing session can be changed from the
  viewer by the appropriate gesture (From \texttt{Skim} this is
  \texttt{Shift-Cmd-click} on one of the pdf lines).
\end{enumerate}

\hypertarget{installation}{%
\subsection{Installation}\label{installation}}

The \texttt{AppleRed.app} directory can be installed either on
\texttt{OS/X} or on \texttt{Linux} by copying it to your local
applications folder: \texttt{\textasciitilde{}/Applications}.\footnote{On
OS/X you could also copy it to the system applications folder.}
Before use it is essential to copy the profile files \verb/.AppleRed*.profile/
from
\begin{verbatim}
        AppleRed.app/Contents/Resources/Profiles/
\end{verbatim}
to your \texttt{\$(HOME)} and, if necessary, adjust them to your
own requirements.

The profiles are nearly identical -- the important things to change are
the PATHs exported from \texttt{.AppleRed-application.profile}; the
other profiles are (but needn't be) essentially the same. The following
command will do the copy:

\begin{verbatim}
cp -pr $(find ~/Applications/AppleRed.app -name ".AppleRed*.profile") ~
\end{verbatim}

At this point it is appropriate   to copy the \texttt{red} and \texttt{cred} scripts, and
their friends to a folder somewhere on your shell; or to make symbolic links to them
from such a folder. These scripts will be found in the folder:\footnote{A set of scripts
is provided that function equally well on Linux and OS/X.}
\begin{verbatim}
        AppleRed.app/Contents//Resources/Shellscripts
\end{verbatim}

\subsection{Invocation}\label{invocation}
\subsubsection{Scripts}\label{scripts}

The \texttt{red} \emph{paths}, and \texttt{cred} \emph{paths}
commands work on both operating systems, independently of whether
the \texttt{AppleRed} app has been started. The former starts the
editing server (if it hasn't already been started) which then starts
editing sessions for the documents denoted in the filestore by the
\emph{paths}. The server runs until the last of its editing sessions
is closed; but keeps running in this case if it was started with
an empty \emph{paths}.

The latter starts its editing sessions independently of any server. It
terminates when its last editing session is closed.

\subsubsection{The AppleRed app}\label{theappleredapp}
On \texttt{OS/X} the app acts as a drop-target, or as an ``open-with''
subject. When invoked, it starts a editing server (if necessary), then
passes its argument to it, then (finally) terminates. Think of it as a
``start button'' for the server and don't expect it to hang around
between ``starts''. Drop the \texttt{AppleRed.app} icon onto the
\texttt{OS/X} dock if you want to make it easily available.

There are many ways of emulating this sort of functionality on
Linux: almost as many as there are Linux distributions. Find a
geek to tell you how on yours: I made it happen on my own Ubuntu 22.04.

\hypertarget{editing-sessions}{%
\subsection{Editing Sessions}\label{editing-sessions}}

In each session window you will see the main text editing
component in the middle, with three minitext-editing components in a row
above it, and a session status component at the bottom. Figure \ref{Screen1}
is an example; it also shows the position of the editing cursor.

The minitexts have buttons to their left labelled \textbf{(A)}
(\emph{Argument}), \textbf{(F)} (\emph{Find}) and \textbf{(R)}
(\emph{Replace}). Pressing one of these buttons clears the adjacent
minitext.


The \emph{Argument} minitext provides arguments for many of the other
editing actions, including those on the \emph{File} menu which start new
editing sessions, allow the document to be saved in a different file,
and so forth.

The \emph{Find} and \emph{Replace} minitexts provide arguments for the
actions of the same name. The checkbox next to \emph{Find} indicates
whether its content will be interpreted as a regular expression or as a
literal pattern. The same checkbox indicates whether the \emph{Replace}
content will be treated as a substitution for a matching regular
expression or as a literal pattern.

Many of the editing actions described below are bound to keystrokes, and
can also be used on the minitexts.

Red shows which of its (mini-)text windows has the keyboard focus by
showing its cursor in red. A text window with a greyed-out cursor
doesn't have the keyboard focus. Focus usually moves into a text window
as the mouse-cursor moves into it, but occasionally after popup warnings
it's necessary to force the issue by clicking in the desired window.

\hypertarget{editingmodel}{\subsection{Editing Model}\label{theediting-model}}



At any stage, the document in a window (minitext or main) may have a
\emph{selection} -- which is a contiguous region of the document. The
selection is bounded at one end by the \emph{document cursor} -- which
is shown as a thin red rectangle -- and at the other end by the
\emph{mark} -- which is shown as a (thinner) blue rectangle. When the
cursor and mark are in the same position only the cursor is shown, and
the selection (in that document) is effectively empty.

The text of a selection is always shown on a coloured background: grey
if the selection was \emph{definite} and pink if the selection was
\emph{tentative}. Figure \ref{Screen2} shows a session
window with a tentative selection.\footnote{It is likely that
this selection was made when the cursor was moved
to the right of the rightmost curly bracket on
line 275.} See \textsf{Treatment of the Selection} for further
detail.

The document cursor can be moved in small increments by the arrow keys,
or can be placed anywhere in the document by clicking with the left
mouse-button. The mark can be placed anywhere in the document by
clicking with the right mouse button.


After every editing action the position of the document in the editor
window is adjusted automatically so that \emph{the document cursor is
visible.} Between editing actions the position of the document in the
window can be adjusted manually by the scrollwheel, This can make the
document cursor ``disappear'' temporarily.

``Dragging'' the mouse with left button pressed causes a selection to be
started (and then extended as the mouse is dragged). If the mouse is
dragged outside the window, then the position of the document is
adjusted so that it remains visible.

There is a \emph{cut-ring} which holds the last several texts copied
(\texttt{C-C}) or deleted from the document. The most recent such text
is also placed on the operating system's clipboard and the most recent
material placed in the system clipboard by Red or any other application
is accessible and can be \emph{paste}d into the document or
\emph{swap}ped with the current selection by invoking the the
\emph{Paste} (\emph{Swap}) actions. These are usually bound to the keys
\texttt{C-V}, and \texttt{C-B}. The \emph{cut} action deletes the entire
selection; it is usually bound to the key \texttt{C-X}. When two ore
more adjacent parts of the document are deleted (or cut) in successive
actions they are treated as if they had been cut at the same time.

\hypertarget{editing-commands}{\subsection{Editing Commands}\label{editing-commands}}

A comprehensive set of editing actions are built-in, and bound to
a standard set of keys or menu entries by default. Some actions are
available on more than one key or menu entry. The actions described
below are a selection of those provided.

The editor can be customised using the \texttt{redscript} language. This
supports the definition of new actions as well as changes of binding of
actions to keys. The language will be described in a separate section.

Neither AppleRed nor I distinguish between the control-shift and (OS/X)
command-shift keys: they appear as \texttt{C-} below. This is entirely
to preserve my sanity, because I use both Linux and OS/X. Although it
reduces the absolute number of keys it is possible to bind to separate
actions, there are still enough for any (sane) user -- especially one
prepared to use the mouse to navigate or hit a menu button occasionally.

\hypertarget{undo-and-redo}{%
\subsubsection{Undo and Redo}\label{undo-and-redo}}

As editing actions happen their effects are pushed onto the \emph{done
stack} for that session.

\emph{Undo} (\texttt{C-z}) undoes the topmost effect on the \emph{done
stack} and pushes it onto the \emph{undone stack}.

\emph{Redo} (\texttt{C-Z}) re-does the topmost effect on the
\emph{undone stack}; and pops it from that stack.

\emph{Redo} can also be invoked from the redo button at the right
end of the menu bar: it is labelled \emph{n}\texttt{\textgreater{}},
where \emph{n} is the depth of the \emph{undone stack}. \emph{Undo}
can also be invoked from the redo button to the left of the redo
button: it is labelled \emph{n}\texttt{\textless{}}, where \emph{n}
is the depth of the \emph{done stack}. Consecutive character
insertions on the same line (between other actions) ``bunched'',
so that their effect can be undone all at once. Consecutive cuts
or other deletions are also bunched.

When editing actions other than \emph{Undo} or \emph{Redo} happen,
the entire \emph{redo stack} is emptied.\footnote{There is in
principle no bound on the depth of the \emph{done stack} (I have
seen it deeper than \texttt{4k}. The bunching of insertions and
deletions make this feasible; but it is inadvisable to try to
\emph{Undo} too far.}



\hypertarget{cut-and-paste}{%
\subsubsection{Cut and Paste}\label{cut-and-paste}}

\emph{Cut} (\texttt{C-x}, or \texttt{F1}) removes the selection from the
document puts it in the cut-buffer, and adds it to the cut-ring.

\emph{Paste} (\texttt{C-v}, or \texttt{F2}) inserts the content of the
cut-buffer into the document and re-selects it.

\emph{Copy} (\texttt{C-c}, or \texttt{F3}) replaces the cut-buffer with
the selection (if there is a selection).

\emph{SwapSel} (\texttt{C-b}, or \texttt{F4}) exchanges the selection
and the cut-buffer (except that if the selection is null, then the
cut-buffer doesn't change). The replaced text is added to the cut ring
as if it had been deleted.

\emph{SwapCursorAndMark} (\texttt{F5}) does what its name suggests.

\hypertarget{find-and-replace-principles}{%
\subsubsection{Find and Replace
(Principles)}\label{find-and-replace-principles}}

In what follows:

\begin{itemize}
\item
  \emph{following} means ``at a position to the right of or below the
  \emph{editing cursor}.''
\item
  \emph{preceeding} means ``at a position to the left of or above the
  \emph{editing cursor}.''
\item
  \textbf{(F)} means ``the pattern specified by the \emph{Find}
  minitext.''
\item
  \textbf{(R)} means ``the replacement specified by the \emph{Replace}
  minitext.''
\end{itemize}

If the checkbox adjacent to the \emph{Find} minitext is checked then the
find text is interpreted as a (Java-style) regular expression. Otherwise
it is interpreted literally.

When the find text is interpreted as a regular expression, the
replacement text is interpreted as the text obtained (using standard
Java regular-expression substitution rules) by substituting each
instance of \texttt{\$n} in it by the text that matches the \texttt{n}th
bracketed expression in the instance of the pattern that is being
replaced. Otherwise it is interpreted literally.


\begin{itemize}
\item
  \emph{FindDown} (\texttt{C-f}, or \texttt{Numpad-0}) selects the
  nearest \emph{following} instance of \textbf{(F)}
\item
  \emph{FindUp} \texttt{(C-F)}, or \texttt{Shift-Numpad-0}) selects the
  nearest \emph{preceeding} instance of \textbf{(F)}
\item
  \emph{FindSelDown} (\texttt{C-Alt-f}, or \texttt{Alt-Numpad-0}) makes
  the current selection the \emph{Find} pattern, turns off regular
  expression interpretation, and then acts as \emph{FindDown}.
\item
  \emph{FindSelUp} (\texttt{C-Alt-F}, or \texttt{Shift-Alt-Numpad-0})
  makes the current selection the \emph{Find} pattern, then acts as
  \emph{FindUp}.
\item
  \emph{ClearFind} (\texttt{C-Numpad-0}, or \texttt{Button(F)}) clears
  the find minitext, and moves the editing cursor to that text.
\item
  \emph{ClearRepl} (\texttt{C-Numpad-.}, or \texttt{Button(R)}) clears
  the replace minitext, and moves the editing cursor to that text.
\item
  \emph{ClearArg} (\texttt{C-Numpad-Enter}, or \texttt{Button(A)})
  clears the argument minitext, and moves the editing cursor to that
  text.
\end{itemize}

If the current selection is an instance of \textbf{(F)} then:

\begin{itemize}
\item
  \emph{ReplaceDown} \texttt{(C-r)}, \texttt{Numpad-.} saves the current
  selection in the cut-buffer and replaces it with the \textbf{(R)}
  selecting the replacement text and leaving the cursor to the right of
  the mark.
\item
  \emph{ReplaceUp} \texttt{(C-R)}, \texttt{Shift-Numpad-.} saves the
  current selection in the cut-buffer and replaces it with the
  \textbf{(R)} selecting the replacement text and leaving the cursor to
  the left of the mark.
\end{itemize}

\hypertarget{find-and-replace-technique}{%
\subsubsection{Find and Replace
(Technique)}\label{find-and-replace-technique}}

Many editors provide a multiplicity of modes for finding and replacing,
and some include the possibility of ``approving'' replacements. In what
follows we explain our simpler approach.

A way of replacing the next instance of \texttt{"FOO"} with
\texttt{"BAR"}, starting from scratch, without using the mouse to
move between texts is to type:

\begin{verbatim}
    C-Numpad-0 F O O Numpad-0 C-Numpad-. B A R Numpad-.
\end{verbatim}

Because the \emph{Find} and \emph{Repl} minitexts are now set
to the right pattern and replacement, replacing the following
instance just requires the two keystrokes \emph{FindDown} \emph{ReplaceDown},
that is, either:
\begin{alltt}
    Numpad-0 Numpad-.      \textrm{or}         C-f C-r
\end{alltt}

Subsequent instances can be found/replaced with the same keystrokes;
and the same technique can be used ``upwards'', by using \emph{FindUp}
and \emph{ReplaceUp} (the same keys pressed, but shifted).

\begin{itemize}
\item
  \emph{ReplaceAll} (\texttt{(F)-\textgreater{}(R)} on the edit menu)
  replaces (without any interaction) all instances of the \emph{Find}
  minitext in the current selection with the \emph{Repl} minitext, and
  selects the resulting transformed text.

  The original selection is preserved in the cut buffer (and therefore
  in the cut ring), so this action can be undone immediately with
  \emph{SwapSel} if you are immediately overcome with remorse
  afterwards, and by browsing in the cut ring if the remorse takes
  longer to arrive.

  If you want to ``approve'' each replacement interactively, then just
  use \emph{FindDown} \emph{ReplaceDown} in sequence repeatedly, undoing
  any replacement you don't approve of with \emph{SwapSel} (or
  \emph{Undo}).
\end{itemize}

To find/replace earlier instances just type the same actions with
the shift key pressed.\footnote{You may have noticed that the
``upward'' variant of an action is bound to the shifted key that
its ``downward'' variant is bound to.}






\hypertarget{indentation}{%
\subsubsection{Indentation and the \texttt{Tab} key}\label{indentation}}

When there is no selection \texttt{Tab} inserts enough spaces to
move the cursor so that it is a multiple of 8 characters from the left
margin.

When \texttt{Red/Autoindent} is enabled, it is assumed that the
indentation of the text following a newline is to be the same as the
indentation of the current line. This is achieved by inserting spaces
after \texttt{Enter} is typed, and removing spaces to the right of the
cursor if necessary.

When there is a \textit{nonempty selection:}

\begin{itemize}
\item
  \emph{Indent selection} (\texttt{Tab}, \texttt{C-Right}) indents the selected lines by 1
  space. If invoked with the \textbf{alt}-shift pressed, then the
  selected lines are indented by prefixing them with the \textbf{(A)}
  text, and then re-selected.
\item
  \emph{Undent selection} (\texttt{Shift-Tab}, \texttt{C-Left}) (when there is a nonempty
  selection) reduces the indentation of the selected lines by 1 space
  where possible. If invoked with the \textbf{alt}-shift pressed, then
  the selected lines are ``undented'' by the (literal) \textbf{(A)} text
  where possible, and then re-selected.
\end{itemize}

The indent and undent selection actions effectively \emph{cut} the
selection and replace it with the indented (undented) text. A
consecutive sequence of \emph{Indent} or \emph{Undent} actions
is undone by \emph{Undo}.



\hypertarget{treatment-of-the-selection}{%
\subsection{Treatment of the
Selection}\label{treatment-of-the-selection}}

Red offers two modes of treating the selection. Most users will decide
on one of them and stick to it forever. In what follows, a ``tentative''
selection means one made by auto-selecting material between matching
brackets.

Having grown accustomed to the second mode in my homegrown editors for
more than thirty years I was reluctant to adopt the first: but after
implementing the differential treatment of ``tentative'' and
``definite'' selections I changed my mind.

\begin{enumerate}
\def\labelenumi{\arabic{enumi}.}
\item
  \textbf{Typing-cuts-selection:} Typing new material automatically cuts
  a ``definite'' selection into the cut buffer (and adds it to the cut
  ring). This is (almost) the behaviour that most people have come to
  expect of editors; except that it does not cut ``tentative''
  selections.
\item
  \textbf{Typing-removes-mark:} Typing new characters removes the mark,
  and thus deselects the \emph{selection}, but does not delete the
  selected material. In this mode Red does not distinguish between
  ``tentative'' and ``definite'' selections.
\end{enumerate}

The choice of modes is made using the \textbf{Red/Typeover} checkbox. As
with most other preferences its value is preserved between \texttt{Red}
invocations.

Tentative selections are made when a closing bracket (of some kind) is
typed, or when the mouse is clicked just to the left of an opening
bracket or to the right of a closing bracket. In both cases the scope of
the selection is the well-nested bracketed text (if there is one)
between the opening and closing bracket. By well-nested we mean
well-nested with respect to the opening (closing) bracket of a
bracketing pair.

\texttt{Red} builds-in most unicode bracketing pairs in addition to the
obvious ``\{\}'', ``()'', ``{[}{]}'' and can be easily adapted to add
more pairs specified by regular expressions; for example
those that match the built-in Latex and HTML block brackets
described below.\footnote{The adaptation may require thought to be
given to the potential for ambiguity, and for the ambiguity to be
resolved programmatically.}

Any \verb/\begin{/$kind_b$\verb/}/ is considered to match
any \verb/\end{/$kind_e$\verb/}/
no matter whether $kind_b$ and $kind_e$ are the same. Likewise
any \verb/</$kind_b$\verb/ ...>/ and \verb/</$kind_e$\verb|/>|.

Tentative selection behaviours are enabled or disabled by the
\texttt{Red/Select/(\ldots@)\}} and
\texttt{Red/Select/\{\ldots\}} menu entries respectively: their values are
indicated on the menu entries and preserved between \texttt{Red} invocations.

\hypertarget{abbreviations}{%
\subsection{Abbreviations}\label{abbreviations}}

Red can be configured with \emph{abbreviations} for texts that contain
unicode symbols that may not appear directly on the keyboard.

One form of abbreviation is built-in, namely
\texttt{\textbackslash{}u}\emph{xxxx}, where each of the four \emph{x}
is a hex digit. It abbreviates the Unicode glyph with that code.

The action \emph{Abbrev} (\texttt{ESC}) finds the longest abbreviation
that appears just before the cursor, and replaces it with the text it
abbreviates.

The action \emph{Unabbrev} (\texttt{Alt-Shift-ESC}) replaces the
character immediately to the left of the cursor with its unicode
encoding, in the form \texttt{\textbackslash{}u}\emph{xxxx}.

\hypertarget{examples-of-abbreviations}{%
\subsubsection{Examples}\label{examples-of-abbreviations}}

Here are extracts from the (extensive) bindings file
\texttt{symbols.redscript} delivered with Red. Notice that both
unicode glyphs and their numbered Unicode representations in the
form \texttt{\textbackslash{}uXXXX} can be used in abbreviations
and their substitutions.

\begin{verbatim}
abbrev "lnot"       "\u00AC"    # ¬
abbrev  "/\u2203"   "∄"
abbrev "land"       "\u2227"    # ∧
abbrev "/\\"        "\u2227" 
abbrev "=>"         "⇒" 
abbrev "<=>"        "⇔" 
abbrev "lor"        "∨"         # ∨
abbrev "\\/"        "\u2228"
abbrev "cap"        "\u2229" 
abbrev "cup"        "\u222a"  
abbrev "euro"       "€"  
abbrev "pound"      "\u20a4"
\end{verbatim}

Words and phrases can also be abbreviated

\begin{verbatim}
abbrev "lb"         "pound"
abbrev "lbs"        "pounds and pounds"
\end{verbatim}

Further details of binding configurations will (eventually) be found in
an appendix.

%%%%%%%%%%%%%%%%%%%%%%%%
%%%%%%%%%%%%%%%%%%%%%%%%
%%%%%%%%%%%%%%%%%%%%%%%%
\newpage
\appendix
\hypertarget{appendix-bindings-and-abbreviations}{%
\section{APPENDIX: Bindings and
Abbreviations}\label{appendix-bindings-and-abbreviations}}

\hypertarget{redscript}{%
\subsection{RedScript}\label{redscript}}

As the editor is started it reads the file
\texttt{\textasciitilde{}/.red/bindings.redscript}: the role of the
script in this file is to provide information needed by each editing
session. There is a good deal of room for customization here, but the
distributed version of the script provides a decent minimum collection
of features that permit some choices to be made by the user.

\hypertarget{appendix-history-of-applered}{%
\section{\texorpdfstring{APPENDIX: History of
\texttt{AppleRed}}{APPENDIX: History of AppleRed}}\label{appendix-history-of-applered}}

\hypertarget{red-starts-life-as-a-toy}{%
\paragraph{Red starts life as a toy:}\label{red-starts-life-as-a-toy}}

\textbf{Red} started life as a program to use as an exemplar for
teaching first-year Oxford undergraduates a short course in
object-orient programming using \textbf{\texttt{scala}}. It was to be
the basis of a number of practical exercises.

Its editing model was identical to that used in the \textbf{dred} editor
I had built from scratch in \textbf{\texttt{java}} in the early 2000's
and had been using ever since. \textbf{Dred} had developed from a
sequence of editors, all using the same \emph{modeless} editing model
derived from an abstract formal specification I had published in the
late 1970's.

\hypertarget{red-becomes-my-editor-of-choice}{%
\paragraph{Red becomes my editor of
choice:}\label{red-becomes-my-editor-of-choice}}

It wasn't long before I decided that it would be a good idea to make
\textbf{Red} more than a toy; and soon after I started its development
(during the first pandemic lockdown period) it became my editor of
choice. Shortly afterwards I found it straightforward to make it
integrable with \texttt{OS/X} as an app, as well as runnable as an
ordinary \texttt{JVM}-compatible program on \texttt{OS/X},
\texttt{Linux}, and \texttt{Windows}.

My intention was to remove some of its more advanced features for use in
the course, and this is what I did: the result was a simpler editor,
\textbf{jed}, published for my students as the course started in Trinity
term in the spring of 2022. For reasons of pedagogy that editor
incorporated \texttt{undo/redo}.

\hypertarget{red-inherits-undoredo-from-jed}{%
\paragraph{\texorpdfstring{Red inherits \texttt{undo/redo} from
Jed:}{Red inherits undo/redo from Jed:}}\label{red-inherits-undoredo-from-jed}}

The \texttt{undo}/\texttt{redo} feature of \textbf{jed} had not been
implemented in the original \textbf{Red}, because the latter's cut-ring
feature met all my reasonable needs. But as soon as the course was over
(early summer 2022) I started to enhance the original \textbf{Red} with
\texttt{undo}/\texttt{redo}.

\hypertarget{apple-changes-its-security-model}{%
\paragraph{Apple changes its security
model:}\label{apple-changes-its-security-model}}

The next spasm of \textbf{Red}'s development was occasioned by a change
in the \texttt{OS/X} security model for apps. To paraphrase my obituary
for the unitary \texttt{AppleRed\ app} that provided editing sessions
(on OS/X and Linux) through an embedded server as well as providing the
more usual \emph{drag-drop}, and \emph{open-with} functionality:

\begin{longtable}[]{@{}l@{}}
\toprule
\endhead
``On \texttt{Catalina} (versions above 10.14) and its successors
the\tabularnewline
situation is complicated by (welcome, but ferocious)
Apple\tabularnewline
security measures that \emph{forbid} the reception of messages
(of\tabularnewline
any kind) by uncertified apps. At the time of writing I see
no\tabularnewline
reasonable way round these measures (and, believe me, I
have\tabularnewline
looked) that can be taken without paying the \textbf{Apple
Ransom},\tabularnewline
which imposes both a financial and an intellectual burden.
I\tabularnewline
have therefore decided to distribute the program as a
simple\tabularnewline
\texttt{OS/X\ app} that invokes a helper program to provide
editing\tabularnewline
session functionality.''\tabularnewline
\bottomrule
\end{longtable}

But by dint of the simple but legitimate technique (outlined in the
obituary, and explained in the source code) I nevertheless managed to
deliver all functionality of an OS/X app on OS/X.

Note that there had never been an analogous problem caused by security
measures on Linux.

\hypertarget{customization-gives-rise-to-redscript}{%
\paragraph{Customization gives rise to
RedScript:}\label{customization-gives-rise-to-redscript}}

A much more recent development was a dramatic increase in the
sophistication of the notation used for customising the editor.
\texttt{RedScript} is a straightforward language, reminiscent of
\texttt{scheme}, that ``collaborates'' with the editor code to provide
the editing context: the content of menus, and the bindings of keys to
actions. Its facilities are still evolving as its potential is explored.

\end{document}
