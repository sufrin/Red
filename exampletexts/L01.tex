"Numpad.\x006d"
"Numpad.\x006b"


(keys ("Numpad.\x006b" . (command "cut"))
      ("Numpad.\x006d" . (command "paste")) )















%xelatex
%
% the top line tells redpdf to use xelatex not pdflatex

% Print a warning if a character is missing.
\tracinglostchars=2 

\documentclass[11pt,a4paper]{article}
\usepackage{unicode-math}
\usepackage{verbatimstyle}
\verbatimsize{\small}

\defaultfontfeatures{Scale=MatchLowercase}
\setmathfont{STIX Two Math}    % For symbols
\setmainfont{STIX Two Text}    % For text
\setmonofont{DejaVu Sans Mono} % for verbatim

\begin{comment}
% Introduce a new family: not really neeeded
\newfontfamily\symbolfamily{STIX Two Math}[
  BoldFont = *,
  BoldFeatures = {FakeBold = 1.05},
  SlantedFont = *,
  SlantedFeatures = {FakeSlant = 0.25},
  BoldSlantedFont = *,
  BoldSlantedFeatures = {FakeBold = 1.05, FakeSlant = 0.25}
]
% Switch to the symbol family (in text).
\def\§#1{\symbolfamily{#1}}
\end{comment}


% Introduce a new family: not really neeeded
\newfontfamily\dejavu{DejaVu Sans}[
  BoldFont = *,
  BoldFeatures = {FakeBold = 1.05},
  SlantedFont = *,
  SlantedFeatures = {FakeSlant = 0.25},
  BoldSlantedFont = *,
  BoldSlantedFeatures = {FakeBold = 1.05, FakeSlant = 0.25}
]


\def\§#1{${#1}$}

%%%%%%%%%%%%%%%%%%%%%
\usepackage[]{}
%%%%%%%%%%%%%%%%%%%%%
\author{Bernard Sufrin}
\title{Eξπλanation of the Unγoδly\footnote{Being a min-fuss example of the use of
XeLaTeX
with mathematical Unicode symbols}}
\date{\today}
\parindent=0pt
\parskip=\medskipamount
%%%%%%%%%%%%%%%%%%%%%
\begin{document}
\maketitle

The two-point list:

\begin{itemize} \item
“Double inverted commas” with a math symbol ($⇒$) placed between dollars,
thus: \$$⇒$\$, and with a symbolic
interlude \§{∀⇛→∃←} — the ‘argument’ of \backslash§{...} command.

\item
and 
$$
        foo ↦ \mathrm{bar} ∧
        foo ⇒ \mathbf{bar} ∧
        {\frac{\mathbf{→}}{→\mathbf{←}}}
$$
\end{itemize}
was set by
\begin{verbatim}
\begin{itemize} \item
“Double inverted commas” with a math symbol ($⇒$) placed between dollars,
thus: \$$⇒$\$, and with a symbolic
interlude \§{∀⇛→∃←} — the ‘argument’ of \backslash§{...} command.

\item
and 
$$
        foo ↦ \mathrm{bar} ∧
        foo ⇒ \mathbf{bar} ∧
        {\frac{\mathbf{→}}{→\mathbf{←}}}
$$
\end{itemize}
\end{verbatim}
What do other symbols, including «brackets» and  $ ⌈⇑ ↠ look ⇓  ↯ ⌉$ like?


Pretty good $∵$ I'm encouraged!

And, as expected, greeks are ok:  αβγδΑΒΓΔ?

Some symbols, for example, $⟦$ and $⟧$ may be harder to
to edit with in Red, not being really fixed-width in
Red's (current, default) editor
font family; although they measure correctly in
DejaVu Sans Mono.

Some other characters aren't part of the STIX Two Text
or the STIX Two Math fonts used in this document.

\bgroup\dejavu{If you see what I mean ⟦⟦⟦⟦⟦⊗⟦⟦⟦⟦xyzzy⟦⟦⟦} (in DejaVu Sans)\egroup

\verb/If you see what I mean ⟦⟦⟦⟦⟦⊗⟦⟦⟦⟦xyzzy⟦⟦⟦} (in DejaVu Sans Mono)/

If you see what I mean ⟦⟦⟦⟦⟦⊗⟦⟦⟦⟦xyzzy⟦⟦⟦ (in STIX Two Text)

If you see what I mean $⟦⟦⟦⟦⟦⊗⟦⟦⟦⟦xyzzy⟦⟦⟦$ (in STIX Two Math)




Why didn't I start using XeLaTex a very long time ago?

\end{document}



