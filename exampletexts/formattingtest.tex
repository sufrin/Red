
A programmer's knowledge and experience isn't organised along syntactic lines
but in much larger conceptual structures: ``algorithm'', ``data structure'', 
``protocol''. We are quite used to classifying
certain kinds of algorithm using terms such as ``dynamic programming'', 
``greedy'', ``divide and conquer''. These terms offer us a level of
discourse above the level of individual programming language constructs,
and thereby facilitate communication between algorithm designers. Likewise
we are used to classifying {\em information structures} in abstract terms
such as ``set'', ``mapping'', ``injective mapping'' and implementation
techniques for them in terms such as ``B-Tree'', ``Hash Table'' -- the
discourse is again above the level of individual programming language constructs 
(well, usually).

In the late 1980s the ``Design Patterns'' community -- which evolved in the
world of object oriented programming, inspired by the work of  
the architect Christopher Alexander -- began to catalogue
and classify patterns of structure and communication which are 
frequently found  in well-designed systems. Much of the vocabulary
was, then, unfamiliar to an orthodox computing scientist: ``Iterator'',
``Action'', ``Subject-Observer'', ``Visitor'' -- but some of the
concepts and patterns were quite familiar -- either from
their theoretical experience, or because they
had been reading and writing programs which utilised the 
patterns.

People learning OOP often complain that the systems they're working with
use inheritance in convoluted ways, and that the flow of control is hard
to follow. Familiarity with some of the design patterns outlined in
these lectures may help you: both to read existing programs, and to
write better ones yourself.


The original ``Gang of Four'' book (with $C^{+\!+}$ as an expository Albatross\footnote{A metaphor
arising from Coleridge's surreal poem ``The Rime of the Ancient Mariner'' in which the
narrator is punished for bringing bad luck to his crewmates by having the
corpse of an Albatross he killed hung around his neck.}) is 
probably still worth reading if you can find the time. You can still pick
up a Kindle  version for about \pounds 10.00, and a hardback of the first edition
