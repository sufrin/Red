\documentclass{ip3}
\lecture{I: Introduction and Fundamental Concepts}
\begin{document}->

\Section{Opening Remarks}


%%%%%%%%%%%%%%%%%%%%%%%%\Rightarr=>
\begin{foil}[The Context]

\begin{note}[Programming in the Large]
      We take ``Programming in the Large'' to mean: \textit{building
      programs with nontrivial internal interfaces}. Your understanding of
      this depends on your understanding of ``nontrivial interface'', of
      course, but \textit{we} take a nontrivial interface to be an
      interface whose change would result in a significant amount of work
      -- say more than a day's work (at either side of the interface) by
      reasonably skilled programmers.\footnote{Ok! What does
      \textit{reasonably skilled} mean?} Another view, not inconsistent
      with this, is that ``Programming in the Large'' means building
      systems that need the work of more than a handful of programmers if
      they are to start functioning in reasonable time.\footnote{Let's be
      reasonable about the meaning of \textit{reasonable}, here!}

\end{note}

\begin{itemize}
        \item Object-Oriented Programming 
        \begin{itemize}
              \item Evolved in response to the challenges of \textit{Programming in the Large}
              \item Well-adapted to most scales of programming task
        \end{itemize}
        
        \vitem Large Systems: some big challenges
        \begin{itemize}
                \item Problem domains can be intrinsically complex 
                \item End-user requirements evolve continuously
                \item Systems evolve over several years or decades
                \begin{note}[Systems Evolve]
                      System designers rarely know in advance the
                      real requirements of their clients or the
                      end-users of their systems.
                      So systems evolve over time as the needs
                      of the end-users become better understood, and
                      as tools and technologies change or obsolesce.
                \end{note}
                \item Underlying hardware \& software technologies change continuously
                      \begin{note}[Legacy Interfaces to Legacy Code]
                      Programmers have to build systems which use
                      legacy interfaces to legacy code, and to
                      cope with the consequences of other people's
                      irreversible design decisions.
                      \end{note}
                \item Systems have to adapt to changing environments
                \item Systems accumulate more and more components
                \item The programming team no longer controls the whole program
                \begin{note}[Code Ownership]
                      The realistic programmer does not expect to
                      possess (let alone \textit{control}) all the source
                      code which contributes to the behaviour of
                      the program under construction.

                      Most programming teams under most circumstances
                      don't write every line of code that is used
                      in their program. They incorporate library
                      components and/or framework code in their systems,
                      rather than reinventing the wheel.

                      The big challenge here (from the perspective
                      of the users of an imported component) is that
                      if the component doesn't do exactly what the
                      team wants, they have to program round the
                      behaviour they don't like.  As well as being
                      intrinsically irritating, this is dangerous,
                      because the behaviour in question might change
                      with the next release of the library/framework.

                      Much important component design activity is
                      generic -- in the sense that it is the design
                      of frameworks that will be finished by clients,
                      and/or of libraries of components that will
                      be used by them.                      

                      From the perspective of the component builder 
                      the challenge is to build functionally stable
                      and well-documented components that are
                      adaptable for use in large numbers of systems.
                \end{note}
        \end{itemize}
\end{itemize}
\end{foil}

%%%%%%%%%%%%%%%%%%%%%%%%%%%%%%%%%%%%%%%%%%%
\begin{foil}[Desirable qualities of a system, a program or a component]
\begin{itemize}
\begin{note}[Specifications and Requirements]
      If a program/component doesn't do what it's supposed to do
      then its other qualities matter little. 
      
      But the specification of a \textit{system}  or a \textit{program}
      is more than a purely technical question. The program, as
      specified, must meet its operational requirements. It must
      support its end users to do their jobs or live their lives.
      
      Those who commission or purchase a corporate or governmental system
      are not often its end users, and there are countless cases of
      expensive systems that, whilst \textit{technically correct} 
      do not meet the requirements of the situation in which they will
      be deployed. The sociotechnical problem of \textit{requirements elicitation} is
      interesting in its own right. 
\end{note}
\item Correctness: perform its task as defined by its specification.
\item Verifiability: the ease with which correctness can be shown 
\begin{note}[Verification, testing and other challenges]
      It can be quite hard to specify software in a way that
      makes its designer's intentions clear whilst implementing it
      by means that make its correctness \textit{provable}. 
\par      
      A slightly weaker form of verification than proof is \textit{testing},
      and these days there are some interesting tools and disciplines 
      that can be used for \textit{systematic} testing --
      particularly of components.
\par      
      It is also a challenge to take a (possibly-prealgorithmic) specification
      and test its potential utility to end-users without first building it, or at least 
      building a prototype.
\end{note}
\item Robustness: the ability of a software system to function without catastrophic failure -- even in abnormal conditions.
\item Maintainability: the ease with which a software system may be adapted to changes in its
specification and/or operational environment.
\item Reusability: the ease with which a component may be used in the construction of a variety of systems.
\item Composability: the ease with which a component can be combined with other components.
\item Efficiency: degree to which space and computational resources are conserved
\item Ease of Use
\begin{note}[Ease of Use]
Of course ease of use, the degree to which brain-cells are conserved
when deploying or using a program, is an extremely desirable quality.

The listed qualities are not necessarily consistent with each
other. For example, often code that is very computationally efficient can be rather
hard to verify. That's why disciplines of program refinement have evolved,
that enable programmers to transform an ``inefficient but obviously 
correct'' program into an ``efficient but not obviously correct'' program
by means of small, understandable, correctness-preserving transformations.

\end{note}
\end{itemize}
\end{foil}


%%%%%%%%%%%%%%%%%%%%%%%%%%%%%%%%%%%%%%%%%%%
\begin{foil}[Managing Complexity]
Managing complexity: the key conceptual tools
 \begin{itemize}
         \vfill\item Modularity
         \begin{itemize}
                 \item complex systems should be composed of modules.
                 \item each component (module) should play a single well-understood role.               
                 \item components should have well-defined interfaces.
          \end{itemize}
          \vfill\item Abstraction (make explicit only what is \textit{relevant} to the task)
                  \begin{itemize}                        
                          \item components should be treated as ``black boxes''
                          \item black boxes with identical specifications should be interchangeable
                          \item we must never rely on knowledge of a component's implementation details
                  \end{itemize}                
          \vfill\item Reuseability
          \begin{itemize}
                 \item Reuse existing components (unless there's
                 good reason to design new ones)
                 \item Design new components for reuseability
                 (unless there's good reason not to)
          \end{itemize}
 \end{itemize}
\begin{note}[Abstraction]
\begin{itemize}
\item Abstraction is our defence against conceptual complexity. We
seek to abstract the essential conceptual core of a problem or
solution from the mass of detail which is not relevant in a
particular context.

\item Modularity -- the decoupling of the implementation structures
of the major conceptual components of a system -- is one of the
most important techniques we can use to defend ourselves against
continuously changing requirements and contexts.
\end{itemize}
\end{note}
\begin{note}[]
The \textit{decoupling} of functional specification from
implementation details is the most important technique which
can be used to defend a program from evolving into a form which
is unnecessarily complicated. The recognition that language
support for such decoupling is necessary came more than fifty
years ago. 

The theories (of program and data refinement) which provide
the conceptual foundation for practising such separation
(at least on a small scale, and for sequential programs)
have been well-understood for more than forty years. They have
been institutionalised in a number of systematic approaches
to program construction, some of which are backed-up by
tools. 

But, as in all engineering domains, there is a difference
between what we  ``know intuitively''  about the artefacts
we build, and what we can prove (intuitively or formally) about them.

Although it is frequently the case that something that was
``known'' before a proof was started is  discovered to
be wrong in the course of an attempt at proof, the economics of program
construction seem to dictate that a huge proportion of mistakes
be discovered accidentally by end-users (and their victims)
rather than deliberately by programmers.

Although formal proof is arduous and unappealing to many, the use of
abstraction techniques can be very helpful in making
a program's structure more intuitively understandable.
\end{note}
\end{foil}

%%%%%%%%%%%%%%%%%%%%%%%%%%%%%%%%%%%%%%%%%%%%%%%%%%%%%%%%%%%%%%%%%%%%
\begin{foil}[Good practice]

\begin{itemize}
\item[]
\begin{itemize}

\item Structure a system or program around whole data structures
      together with their operations (not just around individual operation/function
      definitions).
      \begin{note} 
          A whole data structure usually maps to a \textit{conceptual
          structure} {or a real life entity} in the problem- or
          solution- domain: an individual operation generally
          represents nothing on its own.
\par          
          The advent of type-classes in (some) functional languages 
          also makes this possible -- at least in principle.
      \end{note}



\item Clients that \textit{use} a data structure must depend only on
      its ``official'' interface -- rather than exploiting any internal
      details of its implementation that happen to be known.
       \begin{note}[Clients and Suppliers]
          We say that a piece of code that uses one or more resources supplied by another
          piece of code is a \textit{client} of that \textit{supplier}.
       \end{note}
\end{itemize}


\item[] Object-oriented languages provide linguistic machinery to support and enforce these practices.
      \begin{note}[Black-Box design techniques]
      In engineering these practices are known as ``black-box'' design techniques. 
      In the programming milieu they have various aliases, principally:
      \begin{itemize}
              \item[] Information hiding
              \item[] Abstract (data) types
              \item[] Abstract machines
              There is a technical difference between the ADT and  AM terminology.
              \par
              An abstract machine is \textit{mutable}: it has an internal
              state which can be changed in some externally observable
              way, by its operations.
              \par
              On the other hand, a (value of an) abstract data type
              is \textit{immutable} (doesn't change). 
      \end{itemize}
      \par      
      The logistical value of these practices can be seen when the code
      that implements a structure has to be maintained. For example,
      perhaps the representation of a table that implements a mapping
      has to be changed from a sorted array to a hashtable in order to
      make key lookup faster. If the table implementation wasn't a black
      box then there may be places in the code that uses the table that
      rely on the fact that it is stored as an ordered array. The job
      of maintaining a single data structure now has to be supplemented
      with a search through every program that might use that structure
      for the places where it is used.      
      \end{note}
\end{itemize}
\end{foil}

%%%%%%%%%%%%%%%%%%%%%%%%%%%%%%%%%%%%%%%%%%%%%%%%%%%%%%%%%%%%%%%%%%%%
\begin{foil}[Fundamental Concepts: Objects, Classes, Traits]
\begin{itemize}
\item An \textit{Object} is a record-like data structure, with 
      \begin{itemize}
      \item Fields  -- these are named values, or named variables with values
            \begin{note}
                  The values of fields may be simple scalar values
                  (integers, booleans, characters, floating point numbers,
                  \textit{etc}), or references to other objects)
            \end{note}
      
      \item[]and  
      
      
      \item Methods -- these are procedures (specifications of computations): which may be 
           \begin{itemize}
           \item Operations/Mutators  -- that change the values of the variable  fields
           
           \item[]or
           
           
           \item Functions/Observations -- that compute values from the fields
            
           \end{itemize}
           \begin{note}[Operations and Observations]
               We use the terms \textit{operation} and  \textit{mutator} interchangeably to
               describe a method which may change the state of an object, but which
               does not return any value. We use the terms \textit{observation}
               and \textit{function} interchangeably to describe a method which
               calculates and returns a value or another object. Some functions
               can change the state of an object, but it is bad practice to design
               them to do so in a way which is detectable from outside the object
           \par    
               In practice we shall see that different kinds of client are permitted to invoke
               different kinds of method. The degree of support for classifying
               clients varies dramatically from OOPL to OOPL. The two
               extremes are Eiffel (where arbitrary classes of client can be
               distinguished) and Python (where there are simply users and descendants).
          \par     
               In a non-technical sense it makes sense to let a trusted client have access to 
               some methods which it might be inadvisable to allow an untrusted client to
               use. Trust may offer a performance advantage to the trustee (for example, the use
               of methods which don't check their arguments for soundness), but
               imposes obligations on the trustee, principally to maintain the internal
               invariants of the data structure.
           \end{note}
           
           \item  Fields and methods are collectively called  
                  \textit{members} or \textit{features}.
      \end{itemize}
      \begin{note}
          Most OOPls require that \textit{immutable} composite values
          (for example complex numbers, cartesian co-ordinates, strings)
          be represented as objects, specified by classes or traits. The
          values of the observations one can make of such composite values
          never changes -- although there is no formal prohibition on the details
          of their internal representations changing.  Internal representation
          changes that do not compromise the \textit{specified} behaviour
          are sometimes called ``benign side-effects''.
          \par
          Scala provides abbreviations for designating and constructing certain kinds of built-in 
          immutable composites, notably $n$-tuples (cartesian products).\footnote{For each $n\in\{0..22\}$}
          and functions.
      \end{note}

\end{itemize}
\end{foil}

%%%%%%%%%%%%%%%%%%%%%%%%%%%%%%%%%%%%%%%%%%%
\begin{foil}
\begin{itemize}
\vitem A \textit{Class} is the definition of a family of objects which have features in common
\vitem An \textit{Abstract Class} is a class in which certain features have yet to be fully defined
\vitem A \textit{Trait} (in Scala) is the definition of a family of classes: an interface to all their objects
\end{itemize}
\end{foil}

%%%%%%%%%%%%%%%%%%%%%%%%%%%%%%%%%%%%%%%%%%%
\begin{foil}[Caveat]
\begin{itemize}


\item Some principles and ideas are best discussed first in small case studies

\vitem The practicals will demand work with at least one big program

\vitem Scala is the \textit{medium of instruction}, not the subject of the course
\begin{itemize}
\item It is a complex language with lots of details to understand
\item Some important details are discussed here only in passing
\item You will need to pick up many details from the recommended documentation
\end{itemize}
\end{itemize}
\end{foil}



\end{document}



